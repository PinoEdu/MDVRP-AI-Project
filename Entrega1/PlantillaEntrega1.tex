\documentclass[letter, 10pt]{article}
\usepackage{booktabs}
\usepackage{tabularx}
%%\usepackage[latin1]{inputenc}
\usepackage[spanish]{babel}
\usepackage{amsfonts}
\usepackage{amsmath}
\usepackage[dvips]{graphicx}
\usepackage{url}
\usepackage[top=3cm,bottom=3cm,left=3.5cm,right=3.5cm,footskip=1.5cm,headheight=1.5cm,headsep=.5cm,textheight=3cm]{geometry}



\begin{document}
\title{Inteligencia Artificial \\ \begin{Large}Estado del Arte: Multi-Depot Vehicle Routing Problem\end{Large}}
\author{Eduardo Mauricio Pino Huentelaf}
\date{\today}
\maketitle


%--------------------No borrar esta secci\'on--------------------------------%
\section*{Evaluación}

\begin{tabular}{ll}
Resumen (5\%): & \underline{\hspace{2cm}} \\
Introducci\'on (5\%):  & \underline{\hspace{2cm}} \\
Definici\'on del Problema (10\%):  & \underline{\hspace{2cm}} \\
Estado del Arte (35\%):  & \underline{\hspace{2cm}} \\
Modelo Matem\'atico (20\%): &  \underline{\hspace{2cm}}\\
Conclusiones (20\%): &  \underline{\hspace{2cm}}\\
Bibliograf\'ia (5\%): & \underline{\hspace{2cm}}\\
 &  \\
\textbf{Nota Final (100\%)}:   & \underline{\hspace{2cm}}
\end{tabular}
%---------------------------------------------------------------------------%
\vspace{2cm}


\begin{abstract}
El documento se centra en el Problema de Ruta de Vehículos con Múltiples Depósitos, que implica asignar vehículos desde diferentes depósitos a clientes para minimizar la distancia total recorrida. Se revisan diversas técnicas de optimización, desde métodos clásicos hasta enfoques modernos como algoritmos genéticos. Se presenta un detallado Modelo Matemático para abordar el problema, concluyendo con una síntesis de los hallazgos clave y sugerencias para futuras investigaciones.
\end{abstract}

\section{Introducción}

El \textit{Multi-Depot Vehicle Routing Problem}~\cite{MultiDepotVehicleRouting}, que es una extensión del clásico \textit{Vehicle Routing Problem}~\cite{TruckDispatchingProblem}, implica asignar múltiples vehículos desde diferentes depósitos a un conjunto de clientes, con el fin de minimizar la distancia total recorrida. Este problema tiene aplicaciones en la gestión de flotas de vehículos y la distribución de mercancías, entre otros campos, lo que lo convierte en un tema relevante y de interés.

Se busca abordar este desafío al proporcionar una revisión detallada de las técnicas utilizadas para resolver \textit{MDVRP}. Se espera que al entender estas técnicas, tanto profesionales como investigadores puedan tomar decisiones más fundamentadas sobre qué métodos de optimización emplear, adaptándolos a las necesidades específicas de su contexto. Además, se pretende impulsar la innovación en este campo al identificar áreas de investigación futura y posibles direcciones para el desarrollo de nuevas estrategias y enfoques.

Comienza al proporcionar una explicación detallada de \textit{MDVRP}, destacando su importancia en la logística empresarial. Se exploran las técnicas más relevantes utilizadas para abordar \textit{MDVRP}, desde métodos clásicos hasta enfoques modernos como los algoritmos genéticos. Luego, se presenta el Modelo Matemático utilizado para resolver el problema, proporcionando una comprensión clara de sus componentes esenciales. Finalmente, el informe concluye resumiendo los hallazgos clave y sugiere áreas de investigación futura en el campo del \textit{MDVRP}.

\section{Definición del Problema}

El problema del Enrutamiento de Vehículos con Múltiples Depósitos (\textit{MDVRP}) se remonta a 1959, cuando se planteó por primera vez como \textit{``The Truck Dispatching Problem''}~\cite{TruckDispatchingProblem}. En ese entonces, la necesidad de optimizar las rutas de una flota de camiones encargados de la entrega de gasolina desde un terminal hacia varias estaciones de servicio marcó el inicio de este desafío logístico. El objetivo primordial era asignar eficientemente estaciones a los camiones, garantizando la satisfacción de las demandas de cada estación y minimizando la distancia total recorrida por la flota.

A medida que el problema evolucionó con el tiempo, surgieron diversas dificultades relacionadas con la complejidad combinatoria y la optimización de recursos. Coordinar múltiples vehículos que parten de diferentes depósitos para atender a numerosos clientes resultó ser un desafío de asignación y secuenciación complejo. Además, factores como las capacidades de carga de los vehículos y las ventanas de tiempo~\cite{MDVRPTW} en las que los clientes podían ser atendidos añadieron capas adicionales de complejidad al problema.

En el MDVRP, las variables principales incluyen la asignación de vehículos a depósitos, la secuencia de visitas a los clientes y las rutas que los vehículos siguen para completar las entregas. Estas variables están sujetas a diversas restricciones, como la capacidad de carga de los vehículos y las ventanas de tiempo de los clientes.

Los objetivos del \textit{MDVRP} son claros: optimizar la utilización de recursos, minimizar los costos operativos totales y garantizar la entrega oportuna de bienes a los clientes. Esto implica encontrar soluciones que maximicen la eficiencia de las rutas de entrega y \textbf{minimicen los tiempos de viaje} y los \textbf{costos asociados}.

A lo largo del tiempo, han surgido varias variantes del \textit{MDVRP}, cada una con sus propias características y desafíos únicos. Desde el \textit{MDVRP} con \textbf{ventanas de tiempo} hasta el \textit{MDVRP} con \textbf{capacidades de vehículo heterogéneas}, estas variantes reflejan la diversidad de situaciones encontradas en el ámbito de la logística y el transporte.

En resumen, el \textit{MDVRP} es un \textbf{problema complejo} y \textbf{multidimensional} que requiere enfoques innovadores para su resolución. Su estudio continuo y la búsqueda de soluciones eficientes son fundamentales para mejorar la eficiencia y la rentabilidad en una variedad de contextos industriales.

\section{Estado del Arte}

El Multi-Depot Vehicle Routing Problem (MDVRP) se remonta a 1959 con la formulación de \textit{``The Truck Dispatching Problem''}~\cite{TruckDispatchingProblem}. En este contexto, se abordaba la optimización de las rutas de una flota de camiones encargados de la entrega de gasolina desde un terminal de almacenamiento hacia un extenso número de estaciones de servicio abastecidas por dicho terminal. El objetivo principal era asignar eficientemente estaciones a camiones, asegurando la satisfacción de las demandas de cada estación y minimizando la distancia total recorrida por la flota.

El método propuesto en este inicio se basaba en una formulación de programación lineal y se dividía en múltiples etapas de agregación, en las cuales se llevaban a cabo sub-optimizaciones en pares de puntos o grupos. La determinación del número de etapas de agregación se realizaba a partir de la relación entre la capacidad de los camiones y las demandas de las estaciones.

El procedimiento computacional implicaba una resolución iterativa del problema, comenzando con una asignación inicial de estaciones a camiones y posteriormente realizando correcciones rápidas para mejorar la solución. Para este fin, se empleaban funciones delta para decidir qué entradas no básicas aceptar en cada iteración. Además, se discutía cómo manejar soluciones fraccionarias que pudieran surgir durante el proceso de optimización.

Este problema inicial evolucionó con el tiempo, dando lugar a variantes que consideraban problemas con múltiples productos en cada estación y la capacidad variable de los camiones. Esta diversificación reflejaba la creciente complejidad y variedad de situaciones encontradas en el ámbito del transporte y la logística.
\\

Ya con el paso de los años, comenzaron a implementarse \textit{``Vehicle Routing Algorithms''}~\cite{ImplementingVRAlgorithms}, específicamente en el contexto de la distribución de periódicos en áreas urbanas en el año 1977. La necesidad de mejorar la eficiencia en la entrega de periódicos a cientos de puntos de demanda en una ciudad se debe a varios factores que impactan la logística de distribución. Algunas de las razones de importancia para optimizar este proceso son: reducción de costos en la distribución, mejora de la eficiencia operativa en el menor tiempo posible y reducción de emisiones e impacto ambiental.

La eficiencia en el procesamiento de datos es esencial para garantizar tiempos de ejecución razonables y obtener resultados óptimos en la optimización de rutas de distribución. Por lo tanto, usar \textbf{algoritmos heurísticos} para abordar estos problemas con conjuntos masivos de datos será primordial, permitiendo encontrar soluciones aproximadas en un tiempo razonable.

Esto irá acompañado de utilizar \textbf{estructuras de datos adecuadas}, como los \textbf{árboles o grafos}, lo que puede mejorar significativamente la eficiencia en la manipulación de los datos. Organizar la información de manera que las operaciones de búsqueda, inserción y eliminación se realicen de forma eficiente es fundamental.

Por otro lado, la capacidad de manipular y operar conjuntos masivos de datos de manera efectiva, \textbf{tomando decisiones informadas basadas en análisis detallados de la información disponible}, es crucial para obtener resultados precisos en la optimización de rutas de distribución y en la toma de decisiones logísticas. Esta combinación de algoritmos heurísticos eficientes y estructuras de datos adecuadas sienta las bases para abordar con éxito los desafíos logísticos en el ámbito del \textit{Vehicle Routing} a gran escala.
\\

Luego, en la década de 1980, los investigadores Gilbert Laporte, Yves Nobert y Serge Taillefer se embarcaron en un desafío apasionante~\cite{SolvingAFamilyOfMultiDepot}. En medio de los avances tecnológicos y la creciente demanda de soluciones eficientes, estos investigadores se sumergieron en la complejidad de problemas de enrutamiento asimétricos en entornos \textit{Multi-Depot}.

Utilizando un método de árbol de ramificación y poda, transformaron estos desafíos en problemas de asignación restringida, buscando incansablemente soluciones óptimas que mejoraran la planificación logística y redujeran costos operativos. En un contexto donde la logística es la columna vertebral de la cadena de suministro, este estudio representa un hito en la investigación de optimización de rutas y ubicaciones.

La capacidad para resolver problemas que involucran hasta 80 nodos es destacada. Laporte, Nobert y Taillefer ofrecieron un método de árbol de ramificación y poda, permitiendo resolver problemas de tamaño considerable, lo que demuestra la eficacia y escalabilidad del enfoque propuesto. Su alcance se destaca en su aplicabilidad en entornos logísticos complejos, donde la optimización de rutas, vehículos y ubicaciones de depósitos es crucial.

Al poder abordar problemas de enrutamiento asimétricos en entornos multi-depósito, esta metodología ofrece una solución valiosa para empresas y organizaciones.
\\

Una década después, en 1992, Jorge Orestes Cerdeira realizó una formulación matemática detallada del \textit{Multi-Depot Vehicle Routing Problem}~\cite{MultiDepotVehicleRouting}, donde se define un grafo ponderado no dirigido $G = (V, E, c)$, con $V$ representando el conjunto de nodos (clientes y depósitos), $E$ como el conjunto de aristas con costos asociados, y $c$ como la función de costo de viajar entre nodos. Se establecen restricciones sobre cómo asignar vehículos a depósitos, cómo incluir nodos en ciclos específicos, y cómo minimizar la suma de los costos de las aristas en los ciclos determinados, lo que hace que este problema presente una complejidad \textit{NP-Hard} debido a su naturaleza combinatoria y la presencia de múltiples depósitos y clientes, requiriendo así el desarrollo de algoritmos eficientes y heurísticas para su resolución efectiva.

Estos algoritmos se basan en técnicas que involucran la cobertura de nodos mediante \textbf{\textit{trees and matchings}}, lo que permite establecer límites inferiores en el costo de soluciones óptimas. Además, presentan algoritmos de tiempo polinómico que garantizan que los costos de \textbf{las soluciones obtenidas no excedan el doble de los costos óptimos}, lo cual es fundamental para la optimización de las rutas de vehículos desde múltiples depósitos hacia los clientes. Se llevan a cabo pruebas utilizando grafos de distintos tamaños y se consideran matrices de costos que cumplen con la desigualdad triangular, simplificando así la resolución del problema. Se comparan los límites inferiores obtenidos mediante un método de subgradiente~\cite{ValidationOfSubgradient} con los costos de las soluciones generadas por el algoritmo propuesto por \textit{J.O. Cerdeira}, lo que permite evaluar la eficiencia de los enfoques utilizados en la optimización del \textit{MDVRP}.
\\

Un par de años más tarde, Jacques Renaud, Gilbert Laporte y Fayez Boctor propusieron un enfoque innovador para resolver el MDVRP, utilizando una técnica de optimización conocida como \textbf{Tabu Search}~\cite{TabuSearchHeuristic}. Esta estrategia se basa en explorar el espacio de búsqueda moviéndose de una solución a su mejor vecino, incluso si esto implica una disminución en la función objetivo.

El algoritmo propuesto por este trío de investigadores se divide en dos partes principales: la construcción de una solución inicial y la aplicación de \textit{tabu search}.

En la fase de construcción de la solución inicial, cada vértice se asigna a su depósito más cercano y se aplica una heurística de VRP mejorada para el conjunto de vértices de cada depósito. Se utiliza el algoritmo de Petal Mejorado de Renaud~\cite{PetalHeuristic} para generar conjuntos de rutas que pueden ser atendidas por uno o dos vehículos, seleccionando la mejor ruta mediante la resolución de un problema de partición de conjuntos.

La aplicación de \textit{tabu search} consta de tres fases: Mejora Rápida, Intensificación y Diversificación. Cada fase utiliza procedimientos básicos de 1-ruta, 2-ruta y 3-ruta para mejorar las soluciones. Por ejemplo, el procedimiento 1-ruta se utiliza como post-optimizador en rutas de un solo vehículo, aplicando un mecanismo de mejora 4-opt*~\cite{ComputerSolutions} para un circuito Hamiltoniano.

Los resultados de aplicar \textit{tabu search} en 23 instancias mostraron que supera a otras heurísticas existentes en términos de calidad de las soluciones encontradas. Esto respalda la eficacia y robustez del enfoque \textit{tabu search} para resolver \textit{MDVRP}, destacando su capacidad para mejorar continuamente las soluciones a lo largo de las diferentes fases del algoritmo.
\\

Con el paso de los años, comenzaron a aparecer variantes del \textit{MDVRP}. Al igual que el \textit{MDVRP} surge de \textit{VRP}, en el año 2002 surgió \textit{Multi-Depot Vehicle Routing Problem with Time Windows}~\cite{MDVRPTW}. Donde ahora al problema se le agrega la restricción que cada cliente tiene su horario específico en el que pueden recibir la entrega, lo que agrega una capa adicional de complejidad a la planificación de rutas. Además, los gerentes logísticos se enfrentan al desafío de asignar clientes a los depósitos más cercanos, considerando las capacidades del vehículo y respetando las ventanas de tiempo de los clientes.

Para abordar esta nueva variante del problema original, se recurren a seis heurísticas para la asignación de clientes a depósitos:

Comenzando con la heurística \textbf{\textit{Cluster First Route Second}}, esta estrategia agrupa los clientes por depósito antes de diseñar las rutas de los vehículos. Su principal ventaja radica en mejorar la eficiencia en la asignación de clientes y en la planificación de rutas. Sin embargo, una posible desventaja es que puede generar agrupamientos subóptimos dependiendo de la estrategia de agrupación utilizada.

En cuanto a la heurística \textbf{\textit{Random Assignment}}, esta asigna aleatoriamente a los clientes a los depósitos sin considerar criterios específicos. Si bien es simple de implementar y entender, puede producir asignaciones aleatorias ineficientes.

Por otro lado, la heurística \textbf{\textit{Greedy Assignment}} se basa en un enfoque codicioso para asignar clientes a depósitos, con el objetivo de maximizar algún criterio de optimización, como la minimización de la distancia total recorrida por los vehículos. Aunque es rápida y fácil de implementar, puede quedar atrapada en óptimos locales, lo que podría ser una desventaja.

La estrategia \textbf{\textit{Saving Assignment}} utiliza el concepto de ahorro de distancia para buscar asignaciones que generen reducciones significativas en la distancia total recorrida por los vehículos, lo que optimiza la eficiencia de las rutas. Sin embargo, requiere cálculos adicionales para determinar los ahorros de distancia, lo que podría ser una desventaja.

Por su parte, la heurística \textbf{\textit{Nearest Depot Assignment}} asigna a cada cliente al depósito más cercano en términos de distancia, minimizando los tiempos de viaje y simplificando la planificación de rutas. Aunque es efectiva en la minimización de los tiempos de viaje, puede no tener en cuenta otras consideraciones importantes, como la capacidad del depósito.

Finalmente, la heurística \textbf{\textit{Capacity-Constrained Assignment}} tiene en cuenta las capacidades de los depósitos al asignar clientes, garantizando una distribución equilibrada de la carga de trabajo. Sin embargo, puede ser más compleja de implementar debido a la consideración de las capacidades del depósito.

Donde de todas las heurísticas, se observó que las heurísticas que \textbf{asignaban clientes a depósitos a través de urgencias} fueron las que obtuvieron los mejores resultados computacionales. Estas estrategias basadas en urgencias demostraron un desempeño casi tan bueno como las heurísticas que \textbf{asignaban clientes por \textit{clusters}}. Por otro lado, la heurística de \textbf{asignación cíclica} fue identificada como la que obtuvo los peores resultados en términos computacionales.
\\

En los años posteriores, comenzó a ganar popularidad la aplicación de algoritmos genéticos (\textit{GA}) para abordar el \textit{MDVRP}, aproximadamente desde el año 2009. En este contexto, Beatrice Ombuki-Berman y Franklin Hanshar plantearon esta aplicación~\cite{GeneticAlgorithms2009}, con la generación inicial de una población de posibles soluciones de forma aleatoria. Cada una de estas soluciones, representadas como cromosomas, se convierte en un conjunto de rutas mediante un planificador específico. Luego, se evalúa la calidad de cada solución en función de su aptitud, calculando el promedio de aptitud de toda la población.

El proceso evolutivo se inicia una vez que se tiene la generación inicial de población. En esta etapa, los cromosomas son sometidos a operaciones de cruce y selección, siguiendo los principios de los \textit{GAs} convencionales. Además, se introduce una mutación adaptativa entre depósitos para mejorar las asignaciones iniciales de clientes a depósitos. Para la selección de individuos y la posterior reproducción, se utiliza un modelo de selección de torneo con retención de élite, priorizando a los individuos más aptos.

Un aspecto crucial es la aplicación de un operador de cruce específico del problema que garantiza que las soluciones generadas a través de la evolución genética sean todas factibles para el \textit{MDVRP}. El proceso de optimización incluye la selección de padres, la recombinación mediante cruce, la posibilidad de mutación intra-depósito o inter-depósito, y la aceptación de nuevos descendientes en la población en lugar de los padres originales.

\textbf{El proceso evolutivo continúa iterativamente hasta que se alcanza un número predeterminado de generaciones o se cumple una condición de terminación}. Al final, se devuelve la aptitud promedio y la aptitud de la mejor solución encontrada en la población final como resultado del algoritmo genético. Este enfoque permite encontrar soluciones óptimas o cercanas a óptimas, mejorando la eficiencia y calidad de las soluciones obtenidas.

La comparación de los resultados del algoritmo genético propuesto para \textit{MDVRP} mostró que el \textit{GA} fue competitivo y en algunos casos superó a otros enfoques existentes, como el \textit{GenClust GA} y \textit{Tabu Search}. En general, el \textbf{algoritmo genético propuesto se destacó por su eficiencia y calidad de las soluciones}, posicionándolo como una herramienta valiosa para abordar \textit{MDVRP} y resaltando la importancia de la investigación continua en algoritmos genéticos para problemas logísticos complejos.

Para trabajo futuro, se sugiere la necesidad de seguir investigando y desarrollando \textit{GA} específicamente para \textit{MDVRP}. Se plantea la exploración de nuevas estrategias y enfoques en el diseño de algoritmos genéticos, con el objetivo de mejorar la calidad de las soluciones y competir con enfoques basados en \textit{Tabu Search} u otras metaheurísticas.

Además, se menciona la posibilidad de considerar múltiples objetivos de optimización en el diseño de \textit{GA} para abordar de manera más integral los desafíos asociados con la optimización de rutas de vehículos en entornos logísticos complejos.

\section{Modelo Matemático}

Para el desarrollo del modelo matemático, se llevó a cabo una investigación sobre qué modelo matemático podría ser más adecuado para resolver el \textit{MDVRP}. Tânia Rodrigues Pereira Ramos, Maria Isabel Gomes y Ana Paula Barbosa Póvoa realizaron una investigación comparativa de distintos modelos matemáticos y los rendimientos computacionales que obtenían~\cite{MathematicalModel}. De las 4 formulaciones matemáticas, la mejor resultó ser \textbf{\textit{Two-commodity flow formulation}}, por lo que se optó por utilizar su modelo matemático, el cual será descrito a continuación:

\subsection{Constantes}

\subsubsection{Índices}

$i,j$: Índice de nodo
$k$: Índice del vehículo

\subsubsection{Conjuntos}

$\bar{V} = \{1, ...,n+2w\}$: Conjunto de nodos ; $ \overline{V} = V_{c} \cup V_{d} \cup V_{f}$\\

$V_{c} = \{1, ...,n\}$: Subconjunto de nodos de clientes\\

$V_{d} = \{n+1, ..., n+w\}$: Subconjunto de nodos de depósitos reales\\

$V_{f} = \{n+w+1, ..., n+2w\}$: Subconjunto de nodos de depósitos copias\\

$K = K_{1} \cup ... \cup K_{i}$\\

$K_{i}$: Subconjunto de vehículos pertenecientes al depósito $i$

\subsubsection{Parámetros}

$d_{ij}$: Distancia entre los nodos $i$ y $j$\\

$r_{ij}$: Tiempo de viaje desde el nodo $i$ al nodo $j$\\

$Q_{k}$: Capacidad del vehículo $k$\\

$p_{i}$: Demanda del cliente $i$\\

$t_{i}$: Duración del servicio al cliente $i$\\

$T$: Tiempo máximo permitido para una ruta

\subsection{Variables}

$x_{ijk} = \begin{cases}
            1, & \text{si el sitio $j$ se visita inmediatamente después del sitio $i$, por el vehículo $k$} \\
            0, & \text{en caso contrario}
    \end{cases}$
\\
\\

$y_{ijk}$: Variable de flujo, representa la carga del vehículo cuando el vehículo $k$ viaja de $i$ a $j$.
\\
\\

El flujo $y_{jik}$ representa el $k$ espacio vacío del vehículo; por lo tanto $y_{ijk} + y_{jik} = Q_{k}$
\\
\\

$z_{ik} = \begin{cases}
            1, & \text{si el sitio $i$ es visitado por el vehículo $k$} \\
            0, & \text{en caso contrario}
    \end{cases}$

\subsection{Función objetivo}

\begin{equation}
    \label{eq:FuncionObjetivo}
    Min(F) = \frac{1}{2}\sum_{i \in \bar{V}} \sum_{j \in \bar{V}} \sum_{k \in K}{x_{ijk}d_{ij}}
\end{equation}

Minimizar la distancia total recorrida. Desde dos caminos al definir rutas, cada borde de la solución se cuenta dos veces, duplicando la distancia recorrida. Por lo tanto, para identificar la distancia real, la función objetivo tiene que dividir por 2 para eliminar la distancia del segundo camino.

\subsection{Restricciones}

La función objetivo está sujeta a las siguientes restricciones:

\begin{equation}
    \label{eq:firstConstraint}
    \sum_{j \in \bar{V}}{(y_{jik} - y_{ijk})} = 2p_{i}z_{ik} \quad \forall i \in V_{c}, \forall k \in K
\end{equation}

(\ref{eq:firstConstraint}) Establece que la entrada menos la salida de cada cliente es igual a el doble de la demanda de cada cliente.

\begin{equation}
    \label{eq:secondConstraint}
    \sum_{i \in V_{d}} \sum_{j \in V_{c}} \sum_{k}{y_{ijk}} = \sum_{j \in V_{c}}{p_{j}}
\end{equation}

(\ref{eq:secondConstraint}) Asegura que la salida total de depósitos reales sea igual a la demanda total del cliente.

\begin{equation}
    \label{eq:thirdConstraint}
    \sum_{i \in V_{d}} \sum_{j \in V_{c}} \sum_{k}{y_{jik}} \leq \sum_{k}{Q_{k}} - \sum_{j \in V_{c}}{p_{j}}
\end{equation}

(\ref{eq:thirdConstraint})  Dado que la capacidad de los vehículos puede exceder necesidades de los clientes, dejando algunos vehículos sin usar, la restricción  (\ref{eq:thirdConstraint}) asegura que la entrada total de bienes reales depósitos es, como máximo, la capacidad residual del parque de vehículos. La salida total de cada depósito de copias. Corresponde a la capacidad del parque de vehículos, con base en el depósito real correspondiente.

\begin{equation}
    \label{eq:fourthConstraint}
    \sum_{j \in V_{c}}{y_{ijk}} \leq Q_{k} \quad \forall i \in V_{f}, \forall k \in K_{i}
\end{equation}

(\ref{eq:fourthConstraint}) Establece que la salida de cada vehículo perteneciente al depósito de copias $i$ es menor o igual a la salida de ese vehículo capacidad. Si no se utiliza un vehículo, la salida será entonces cero$;$ si se utiliza un vehículo, el flujo de salida igualar la capacidad de ese vehículo.

\begin{equation}
    \label{eq:fifthConstraint}
    \sum_{i \in \bar{V}}{x_{ijk}} = 2z_{jk} \quad \forall j \in V_{c}, \forall k \in K
\end{equation}

(\ref{eq:fifthConstraint}) Garantiza que cualquier solución factible contenga dos aristas incidentes a cada cliente.

\begin{equation}
    \label{eq:sixthConstraint}
    y_{ijk} + y_{jik} = Q_{k}x_{ijk} \quad \forall i \in \bar{V}, \forall j \in \bar{V}, \forall k \in K
\end{equation}

(\ref{eq:sixthConstraint}) Garantiza que el flujo de entrada más el flujo de salida de cualquier nodo sea igual la capacidad del vehículo que visita el nodo.

\begin{equation}
    \label{eq:seventhConstraint}
    \sum_{k \in K}{z_{ik}} = 1 \quad \forall i \in V_{c}
\end{equation}

(\ref{eq:seventhConstraint}) Cada cliente deberá ser visitado por un único vehículo.

\begin{equation}
    \label{eq:eigthConstraint}
    y_{ijk} \leq BigMz_{ik} \quad \forall i \in V_{c}, \forall j \in \bar{V}, \forall k \in K
\end{equation}

(\ref{eq:eigthConstraint}) Pone a cero la variable de flujo  $y_{ijk}$ si cliente $i$ no es visitado por el vehículo $k$.

\begin{equation}
    \label{eq:ninethConstraint}
    \sum_{i \in V_{c}}{t_{i}x_{ijk}} + \sum_{i \in \bar{V}} \sum_{j \in \bar{V}}{r_{ij}x_{ijk}} \leq 2T \quad \forall k \in K
\end{equation}

(\ref{eq:ninethConstraint}) Garantiza que la duración de cada ruta (incluido el servicio y el tiempo de viaje) no excede el tiempo máximo de enrutamiento permitido.

\begin{equation}
    \label{eq:tenthConstraint}
    \sum_{j \in V_{c}}{x_{ijk}} \leq 1 \quad \forall i \in V_{d}, \forall k \in K_{i}
\end{equation}

(\ref{eq:tenthConstraint}) Asegura que cada vehículo salga de su depósito de origen una vez como máximo.

\begin{equation}
    \label{eq:eleventhConstraint}
    \sum_{i \in V_{c}}{x_{ijk}} = 0 \quad \forall j \in V_{f}, \forall k \notin K_{j}
\end{equation}

\begin{equation}
    \label{eq:twelveConstraint}
    \sum_{j \in V_{c}}{x_{ijk}} = 0 \quad \forall i \in V_{d}, \forall k \notin K_{i}
\end{equation}

(\ref{eq:eleventhConstraint}) y (\ref{eq:twelveConstraint}) Garantizan conjuntamente que un vehículo no pueda salir y regresar a un depósito que no sea su domicilio (tanto real como copia).

\begin{equation}
    \label{eq:thirteenthConstraint}
    y_{ijk} \geq 0, x_{ijk} \in \{0, 1\}, z_{ik} \in \{0, 1\} \quad \forall i, j \in \overline{V}, k \in K
\end{equation}

(\ref{eq:thirteenthConstraint}) Dominios de variables se dan en la restricción.

\section{Conclusiones}

En primer lugar, se observa que si bien todas las técnicas revisadas tienen como objetivo resolver el mismo problema fundamental del \textit{MDVRP}, existen diferencias significativas en sus enfoques y metodologías. Desde las primeras formulaciones basadas en \textbf{programación lineal} hasta los enfoques más recientes como los \textbf{algoritmos genéticos}, cada técnica presenta sus propias ventajas y desventajas en términos de eficiencia computacional, calidad de las soluciones obtenidas y capacidad para abordar las complejidades del problema.

En cuanto a las similitudes, \textbf{todas las técnicas revisadas comparten el objetivo común de optimizar las rutas de vehículos desde múltiples depósitos hacia los clientes}, buscando \textbf{minimizar los costos operativos y mejorar la eficiencia de la distribución}. Además, muchas de estas técnicas se basan en \textbf{heurísticas} y \textbf{algoritmos de búsqueda} para encontrar soluciones factibles en un tiempo razonable, dada la complejidad computacional de \textit{MDVRP}.

Por otro lado, las diferencias radican en los enfoques específicos utilizados por cada técnica. Por ejemplo, algunas técnicas se centran en la construcción progresiva de soluciones, mientras que otras se basan en métodos de optimización más avanzados, como el uso de operadores genéticos. Además, algunas técnicas pueden ser más adecuadas para ciertos tipos de instancias del problema, como aquellas con ventanas de tiempo o capacidades variables de vehículos.

En cuanto a las limitaciones, se identifican varios aspectos que pueden \textbf{afectar la efectividad de las técnicas} revisadas. Estas incluyen la capacidad para manejar instancias de \textbf{gran tamaño del problema}, la sensibilidad a la \textbf{calidad de los parámetros de entrada} y la posibilidad de \textbf{quedar atrapado en óptimos locales} en lugar de encontrar soluciones globales óptimas.

En términos de trabajo futuro, se identifican varias áreas de investigación prometedoras. Por ejemplo, la exploración de enfoques distintos para los algoritmos genéticos, podría conducir a mejoras significativas en la calidad de las soluciones obtenidas. Además, la investigación en técnicas de optimización multi-objetivo podría ayudar a abordar de manera más integral los desafíos asociados con la optimización de rutas de vehículos en entornos logísticos complejos.

Para contribuir a futuras investigaciones, se sugiere explorar la aplicación de técnicas de aprendizaje automático y análisis predictivo para mejorar la precisión de los modelos utilizados en la optimización de \textit{MDVRP}. Además, la integración de datos en tiempo real y la consideración de factores dinámicos, como el tráfico y las condiciones climáticas, podrían ayudar a mejorar la robustez y adaptabilidad de los enfoques existentes. En resumen, el campo del \textit{MDVRP} ofrece numerosas oportunidades para la innovación y el avance, y se espera que futuras investigaciones continúen explorando nuevas estrategias y enfoques para resolver este importante problema logístico.

\section{Bibliografía}
\bibliographystyle{plain}
\bibliography{Referencias}

\end{document} 
